\documentclass[12pt, a4paper]{memoir}

% Essential packages
\usepackage[T1]{fontenc}
%%\usepackage[latin1]{inputenc}
\usepackage[french,english]{babel}

% personal packages
\usepackage [includehead, margin=1.5cm]{geometry}                % See geometry.pdf to learn the layout options. There are lots.
\usepackage{graphicx} % include images
\usepackage{wrapfig} % wrap text around figures
\usepackage{hyperref} % urls, hyperlinks, etc
% \usepackage{pdfpages} % include pdf pages
\usepackage{enumitem} % customize itemize
\usepackage[backend=bibtex]{biblatex} % bibliography
\usepackage{csquotes} % Quote bibliography and add hyperlinks
% \usepackage{titlesec} % Modify chapter headings
% \usepackage{wrapfig} % wrap text around figures
% \usepackage{float} % force figure positions at the end of the file
\usepackage[linesnumbered,ruled,vlined]{algorithm2e}

% Packages from uga template
%\usepackage{fullpage}
\usepackage{mathptmx} % font = times
\usepackage{helvet} % font sf = helvetica
% \usepackage{amsmath}
% \usepackage{relsize}
% \usepackage{tikz}
% \usepackage{booktabs}
% \usepackage{textcomp}%textquotesingle
% \usepackage{multirow}
%
\usepackage{minted}

\usepackage{subcaption}
\captionsetup{compatibility=false}
% \usepackage{appendix}
% \usetikzlibrary{arrows,shapes,positioning,shadows,trees}
% \makesavenoteenv{tabular}
% \makesavenoteenv{table}

\def\checkmark{\tikz\fill[scale=0.4](0,.35) -- (.25,0) -- (1,.7) -- (.25,.15) -- cycle;}

%Style des têtes de section, headings, chapitre
\headstyles{komalike}
\nouppercaseheads
\chapterstyle{dash}
\makeevenhead{headings}{\sffamily\thepage}{}{\sffamily\leftmark} 
\makeoddhead{headings}{\sffamily\rightmark}{}{\sffamily\thepage}
\makeoddfoot{plain}{}{}{} % Pages chapitre. 
\makeheadrule{headings}{\textwidth}{\normalrulethickness}
%\renewcommand{\leftmark}{\thechapter ---}
\renewcommand{\chaptername}{\relax}
\renewcommand{\chaptitlefont}{ \sffamily\bfseries \LARGE}
\renewcommand{\chapnumfont}{ \sffamily\bfseries \LARGE}
\setsecnumdepth{subsection}
\addbibresource{biblio.bib}

% Customize itemize item markers
\renewcommand\labelitemi{-}

% Add source to figures caption
\newcommand*{\captionsource}[2]{%
    \caption[{#1}]{%
        #1%
        \\\hspace{\linewidth}%
	\textbf{Source:} \textit{#2}%
    }%
}

% Some settings for the title page
%\titleformat{\chapter}[display]{\normalfont\huge\bfseries}{}{0pt}{\Huge}
%\titlespacing*{\chapter} {0pt}{20pt}{40pt}

% Force footnotes to stay on one page
\interfootnotelinepenalty=10000

% Title page formatting -- do not change!
\newcommand{\HRule}{\rule{\linewidth}{0.5mm}}
\pretitle{\HUGE\sffamily \bfseries\begin{center}\HRule \\[0.2cm]} 
	\posttitle{\end{center}\HRule}
\preauthor{\LARGE  \sffamily \bfseries\begin{center}}
\postauthor{\par\end{center}}
\newcommand{\jury}[1]{% 
\gdef\juryB{#1}} 
\newcommand{\juryB}{} 
\newcommand{\session}[1]{% 
\gdef\sessionB{#1}} 
\newcommand{\sessionB}{} 
\newcommand{\option}[1]{% 
\gdef\optionB{#1}} 
\newcommand{\optionB} {}

\renewcommand{\maketitlehookd}{% 
\vfill{}  \large\par\noindent  
\begin{center}\juryB \bigskip\sessionB\end{center}
\vspace{-1.5cm}}
\renewcommand{\maketitlehooka}{% 
	\vspace{-1.5cm}\noindent\includegraphics[height=10ex]{./imgs/uga-logo.png}\hfill\includegraphics[height=10ex]{./imgs/ryax-logo.png}\hfill\includegraphics[height=10ex]{./imgs/Logo-LIG.jpg}\hfill\includegraphics[height=12ex]{./imgs/ENSIMAG.png}
\bigskip
\begin{center} \large
Master of Science in Informatics at Grenoble \\
Master Informatique \\ 
Specialization \optionB  \end{center}\vfill}
% =======================End of title page formatting

\option{MoSIG} 
\title{Simulation of a Kubernetes Cluster with Validation in Real Conditions} %\\\vspace{-1ex}\rule{10ex}{0.5pt} \\sub-title} 
\author{LARUE Théo}
\date{Defense Date, 2020} % Delete this line to display the current date
\jury{
	Research project performed at Laboratoire d'Informatique de Grenoble \\\medskip
Under the supervision of:\\
Michael Mercier\\\medskip
Defended before a jury composed of:\\
Head of the jury\\
Jury member 1\\
Jury member 2\\
}
\session{September \hfill 2020}
\setcounter{tocdepth}{2}
\setcounter{secnumdepth}{4}

\begin{document}
\selectlanguage{english} % french si rapport en français
\frontmatter
\begin{titlingpage}
\maketitle
\end{titlingpage}

%\small
\setlength{\parskip}{-1pt plus 1pt}

\renewcommand{\abstracttextfont}{\normalfont}
\abstractintoc
\begin{abstract} 
	TODO : rework this with the new intro


	The rise of containerized applications has provided web platforms with
	much more control over their resources than they had before with their
	physical servers. Soon enough, developers realized they could go even
	further by automating container management operations to allow for even
	more scalability. The Cloud Native Computing Foundation was founded in
	this context, and developed Kubernetes which is a piece of software
	capable of container orchestration, or in other words, container
	management. Now, as we observe a convergence between HPC (High
	Performance Computing) and the Big Data field where Kubernetes is
	already the standard for some applications such as Machine Learning,
	discussions about leveraging containers for HPC applications rose and
	interest in Kubernetes has grown in the HPC community. One of the many
	challenges the HPC world has to face is scheduling, which is the act of
	allocating tasks submitted by users on available resources. In order to
	properly evaluate and develop schedulers researchers have used
	simulators for decades to avoid running experiments in real conditions,
	which is costly both in time and resources. However, such simulators do
	not exist for Kubernetes or are not open to the public. While the
	default scheduler works great for most of the Cloud Native
	infrastructures Kubernetes was designed for, some teams of researchers
	would rather be able to experiment with different batch processing
	policies on Kubernetes as they do with traditional HPC. Our goal in
	this master thesis is to describe how we developed Batkube, which it is
	an interface between Kubernetes schedulers and Batsim, a general
	purpose infrastructure simulator based on the Simgrid framework and
	developed at the LIG.

\end{abstract}
\abstractintoc

\renewcommand\abstractname{Acknowledgement}
\begin{abstract}
I would like to express my sincere gratitude to .. for his invaluable assistance and comments in reviewing this report... 
Good luck :) 
\end{abstract}

\renewcommand\abstractname{R\'esum\'e}
\begin{abstract} \selectlanguage{french}
	Abstract mais en franchais
\end{abstract}
\selectlanguage{english}

\tableofcontents* % the asterisk means that the table of contents itself isn't put into the ToC
\normalsize

\mainmatter
\SingleSpace

\chapter{Introduction}

OUTDATED 

TODO: Make another pass on that section after everything else is redacted (or
at least the soa)\\

The need for scalable computing infrastructure has increased tremendously in
the last decades. Nearly every field of computer science, from research to the
service industry, now needs a proper infrastructure and by 2025, computation
technology could reach a fourth of the global electricity
spending\cite{andrae2017total}.  Even the public sector is now in need for
efficient distributed infrastructure as the concept of smart cities is
developing.

Organizations generally know what type of infrastructure will meet their needs.
It can take the form of Big Data centers to store and analyze data,
High-Performance Computers for computing intensive tasks or GPU banks for
machine learning or crypto-currency mining.  However, studying those
infrastructures extensively is much more challenging.  As these computers reach
scales in the order of warehouses\cite{barroso2018datacenter}, quantifying a
system's performance under varying loads, applications, scheduling policies and
system size quickly becomes undoable without expensive real world experiments.
In fact, the nature of scheduling problems\cite{scheduler-complexity} alone
make theoretical studies hard.  This is an issue for organizations as they rely
on thoses studies to determine the size of the required system or choose
optimal scheduling
policies.\\

Simulation allows to tackle these issues by enabling users to draw conclusions
empirically without the need to fire up real workloads. Indeed, running an
entire experimental campaign on a real system represents consequencial costs
both in time and money. With simulation, The gain in both time and spent energy
can be extreme : a HPC job spanning months on a real system can be resolved in
a matter of minutes on any domestic computer.  Another major point is that it
also brings reproducibility to these experiments, that otherwise would have to
be run on the exact same systems as their first iteration. With simulation, one
can recreate the same conditions for any experiment anywhere they want, and
expect the same results.\\

However, simulations need to be run with sound models for the results to be
exploitable and in that regard, simulators usually fall under several
pitfalls\cite{poquet:tel-01757245}. Very often simulators are implemented at
the same time as new schedulers or \textit{Resource and Jobs Management
Systems (RJMS)}\footnote{The RJMS is the software at the core of the cluster. It is a
synonym for a scheduler and manages resources, energy consumption, users' jobs
life-cycle and implements scheduling policies.} in order to validate their
algorithms. Thus, they are strongly coupled together and are not usable with
any other software. They are either shipped with the software itself or worst,
they are never released and discarded at the end of the development process.
Moreover, still according to \cite{poquet:tel-01757245}, strong coupling may
lead to unrealistic models. In that case cluster resources can be accessed with
ease by the scheduler, resulting in it having very precise information about the
system state to take its decisions.  This conflicts with the real world as a
scheduler may not have access to all the information it wants, or may suffer
from latency when getting it from the system.\\

To try and assess these issues a team of researchers at the LIG developed
Batsim\cite{dutot:hal-01333471} which is a general purpose infrastructure
simulator with modularity and separation of concerns in mind. Batsim is based
on SimGrid\cite{casanova:hal-01017319} which is a framework for developing
simulators for distributed computer systems. Simgrid is now a 20 years old
framework that has been used in many
projects\footnote{https://simgrid.org/usages.html}, making it a sound choice to
run scalable and accurate models of the reality.

Batsim was designed to support algorithms written in any languages, as long as
they support its communication protocol. It means that, while any scheduler
found in the wild can potentially be run on a Batsim simulation, they still
have to be adapted to make them compatible. This master's project is dedicated
on developing an interface between Batsim and
Kubernetes\footnote{https://github.com/kubernetes/kubernetes/} schedulers in
order to run Kubernetes clusters simulations. Kube\footnote{Another term to
designate Kubernetes. It is also sometimes called k8s.} is an open source
container management software widely exploited in the industry for its ease of
use and wide range of capabilities. It has freed developers from the cumbersome
task of setting up low level software infrastructure on their servers and
automates maintenance, scaling and administration of their applications. For
all these reasons it has become a de-facto solution for any organization that
wishes to build new internet platforms from the ground up.\\

TODO : what we where able to do (summary of the simulator capabilities,
experimentations, results)



\chapter{Background and related work}

%This section is organized as follow. First we put Kubernetes in context, in
%light of the advances made in web application development. As we will see,
%despite these adavances in the automation of resource management many
%fundamental questions remain that can only be anwsered by extensive studies on
%computer systems. One such question is the problem of scheduling tasks on
%compute resources, which we briefly present. We end this section by presenting
%the concepts of Batsim which is a distributed system simulator especially well
%suited for studies on scheduling alrogithms.

\section{Studying computer infrastructures} \label{study-computing-infra}

Eventhough this paradigm enabled developing new applications with ease, many
questions remain: what type of infrastructure would be best suited for my
application? Would my application benefit from more cpu cores? How would
different scheduling policies affect my application? Would my batch jobs
compute faster with a different topology? To answer these interrogations one
must conduct studies to experiment with different configurations.\\

Studying an entire computing infrastructure is not an easy feat, first because
every infrastructure is unique. There are as many types of infrastructure as
there are use cases, each having a different vision on efficiency and what
metrics are critical to the system: latency, bandwith, resource availability,
computational power or cost effectiveness (the latter boils down to energy
efficiency). This variety of purposes translates to the type of hardware used
and the topology of the infrastructure. Some systems are centralized like HPC
and Data Centers, others are meant to be used from a distance like Cloud
Computing infrastructures and others are decentralized like Grid Computing,
Volunteer Computing and Peer to Peer computing. There are as many systems as
there are objectives to be achieved.

As a consequence, there are no general tools to study those systems.
Furthermore, as the biggest supercomputers are approaching the exascale
barrier\footnote{https://www.top500.org/news/japan-captures-top500-crown-arm-powered-supercomputer/}
and consist of thousands of nodes with millions of cpu cores (more than 7M for
the new ``Fugaku'' Japanese supercomputer), no human would be capable of
building a general mathematical model that would be accurate enough to predict
the behavior of those systems under varying conditions. Also, interactions
between the various components of those systems may lead to unexpected
behavior\cite{10.1007/978-3-319-09873-9_12} that can hardly be predicted.

In order to extensively experiment on a given system, there are 3 options left
as described in\cite{legrand2015scheduling}: \textit{in vivo}, \textit{in
vitro} and \textit{in silico} studies, which correspond respectively to
experiments on real testbeds, emulation and simulation.  

The next parts are mostly built upon \cite{legrand2015scheduling} and
\cite{casanova:hal-01017319} and are aimed at depicting the current landscape
of experimentation on distributed systems.

\subsubsection{\textit{in vivo} and \textit{in vitro} studies}

The most direct approach to study an infrastructure is running \textit{in vivo}
experiments, that is to say running experiments on a real testbed. This will
produce the most accurate results, however it poses major scalability and
reproducibility issues.

Experiments conducted on real systems may prove difficult to reproduce, as one
must have access to the same system to reiterate it. Even then, changes to the
infrastructure hardware and software environment diminish the chances of
getting the same conditions. One solution to this problem is running \textit{in
vitro} studies, that is to say run an emulation of the system (virtual machines
or a network emulation for example). This resolves the issue of
reproducibility, however the matter of the cost in energy and time remains (if
anything, emulation aggravates these costs).

This cost is exacerbated by the many iterations of a same experiment one must
conduct in order to get statistically significant results. Workloads submitted
by real users can last from hours to months and have substantial costs in
energy: the means required to run them are too great and research to optimize
or simply study these systems can not justify this waste of resources. For all
these reasons scientists resort to simulation to study these computing
infrastructures.

\subsubsection{\textit{in silico} studies, or simulation}

Simulation allows scientists to conduct experiments or thought experiments that
would otherwise not be possible in the real world. One can think of simulations
of the universe, prediction models for the weather or modeling some microbiome
in biology. Computing itself makes no exception and researcher have created
models of computer systems in order to experiment with new scheduling policies,
network topologies or planning for systems capacity, for example. Simulation
dramatically reduces experimentation cost and allows for reproducibility.
Workloads on supercomputers may span weeks or months, whereas a single standard
laptop can simulate this same workload in a matter of seconds or hours,
depending on the simulator. More importantly, other scientists would need
access to the same system the experiment was run on to reproduce the experiment
whereas a simulator supposedly brings the same output regardless of the device
it is run on. The only caveat that remains in terms of reproducibility lies in
the application traces used to run off-line simulations that contain critical
data concerning the application it was generated with.

However, even though simulators theoretically allow for effortlessly
reproducible experiments, the way they are developed sometimes make them hardly
usable at all. Indeed, often simulators are one shot programs made to validate
the models of one scientific project and end up discarded once the paper is
redacted, unusable for any other project unmaintained or plainly unreleased.

TODO; finish this paragraph \& introduce SimGrid in this context

\section{SimGrid}

TODO
To understand Batsim's paradigms and view on simulation, we first need to
present Simgrid's paradigms. As the latter is the framework that Batsim builds
upon, Batsim and Simgrid views on simulation cannot be distinguished.

\section{The scheduling problem}

%In particular, this work is targeted at experimenting with scheduling in a
%distributed system driven by Kubernetes. Here we present a general definition
%of scheduling, and the challenges it tackles.

One notorious problem on the field of distributed systems is the allocation of
queued jobs to available resources.

\begin{displayquote}[][]
	\textbf{schedule} \textit{n.} : A plan for
	performing work or achieving an objective, specifying the order and
	allotted time for each part.
\end{displayquote}

In a general way, scheduling is the concept of allocating available resources
to a set of tasks, organizing them in time and space (the resource space). The
resources can be of any nature, and the tasks independent from each others or
linked together.

In computing the definition remains the same, but with automation in mind.
Schedulers are algorithms that take as an input either a pre-defined workload,
which is a set of jobs  to be executed, or single jobs submitted over time by
users in an unpredictable manner (as it is most often the case with HPC for
example). In the latter case, the jobs are added to a queue managed by the
scheduler. Scheduling is also called batch scheduling or batch processing, as
schedulers allocate batches of jobs at a time. Jobs are allocated on machines,
virtual or physical, with the intent of minimizing the total execution time,
equally distributing resources, minimizing wait time for the user or reducing
energy costs. As these objectives often contradict themselves so schedulers have
to implement compromises or focus on what the user requires from the system.

The scheduler has many factors to keep in mind while trying to be as efficient
as possible, such as:

\begin{itemize}
	\item Resource availability and jobs resource requirements
	\item Link between jobs (some are executed in parallel and need synchronization, some are independent)
	\item Latency between compute resources
	\item Compute resources failures
	\item User defined jobs priority
	\item Machine shutdowns and restarts
	\item Data locality
\end{itemize}

All these elements make scheduling a very intricate problem that is at best
polynomial in complexity, and often NP-hard
(\cite{10.1016/S0022-0000(75)80008-0}, \cite{scheduler-complexity}). In order
to better study the effect of different scheduling policies on a system a
reasearcher team at the LIG have created Batsim which is a versatile
distributed system simulator built on SimGrid and focused on the study of
schedulers.

\section{Batsim}

Batsim\cite{dutot:hal-01333471} is a distributed system simulator built upon
the SimGrid framework. Its main objective is to enable the study of RJMS
without the need to implement a custom simulator, by providing a universal text
based interface. \\

It is entirely deterministic so at to make the studies easily reproducible.
Its event-based models will provide the same results given the same inputs and
decision process. One other way Batsim facilitates reproducibility is through
its user-defined inputs. Unlike other HPC or grid computing simulations that
run on existing application traces, Batsim takes a user defined workload as an
input. As a consequence, the user has no concerns such as intellectual property
on application traces and may provide all his experiments materials and
environment.  Another advantage of this system is that the user can adapt the
workload depending on its needs, to achieve different levels of realism.

Batsim, just like SimGrid, aims at being versatile. The common belief is that
specialization is the key to achieving realistic results, however according to
SimGrid this versatility is all but an obstacle to
accuracy\cite{casanova:hal-01017319}: it is on the contrary the key to their
results which are both scalable and accurate. Batsim computation platforms are
SimGrid platforms meaning that theoretically, they may be as broad as SimGrid
allows it. In reality any SimGrid platform is not a correct Batsim platform.
Because Batsim aims at studying RJMS software, it requires a \textbf{master}
node that will host the decision process. The other hosts (or computational
resources) will have either the roles of \textbf{compute\_node} or
\textbf{storage}. Still, the user may study any topology he wishes using
SimGrid models.

Thanks to its own message interface based on Unix sockets, Batsim is language
agnostic which means that any RJMS can be plugged into it as long as it
implements the interface.

TODO: why batsim?

Because it was developed at the lig and they want to expand its capabilities.
It is language agnostic to very convenient to work with.

\section{Kubernetes in the context of Cloud Computing}

In the early stages of application development, organizations used to run their
services on physical servers. With this direct approach came many challenges
that needed to be coped with manually like resources allocation,
maintainability or scalability. In an attempt to automate this process
developers started using virtual machines which enabled them to run their
services regardless of physical infrastructure while having a better control
over resource allocation.  This led to the concept of containers which takes
the idea of encapsulated applications further than plain virtual machines.

\begin{figure}[h]
	\centering
	\includegraphics[width=\textwidth]{./imgs/container_evolution.png}
	\captionsource{Evolution of application deployment.}{https://kubernetes.io/docs/concepts/overview/what-is-kubernetes/}
	\label{fig:container-evolution}
\end{figure}

Containers can be thought of as lightweight virtual machines. Unlike the
latter, containers share the same kernel with the host machine but still allow
for a very controlled environment to run applications. There are many
benefits to this : separating the development from deployment, portability,
easy resource allocation, breaking large services into smaller micro-services
or support of continuous integration tools (containers greatly facilitate
integration tests).\\

The CNCF\footnote{\url{https://www.cncf.io/}} (Cloud Native Computing
Foundation) was founded in the intent of leveraging the container technology
for an overall better web. In a general way, we now speak of these
containerized and modular applications as cloud native computing :

\textit{``Cloud native technologies empower organizations to build and run
	scalable applications in modern, dynamic environments such as public,
	private, and hybrid clouds. Containers, service meshes, microservices,
	immutable infrastructure, and declarative APIs exemplify this
	approach.}

\textit{These techniques enable loosely coupled systems that
	are resilient, manageable, and observable.  Combined with robust
	automation, they allow engineers to make high-impact changes frequently
	and predictably with minimal toil.``}\footnote{\url{https://github.com/cncf/toc/blob/master/DEFINITION.md}}

Kubernetes\footnote{\url{https://kubernetes.io/}} is the implementation of this
general idea and was anounced at the same time as the CNCF. It aims at
automating of the process of deploying, maintaining and scaling containerized
applications. It is industry grade and is now the de-facto solution for
container orchestration.

\section{Related work}

Simulators are often intentionally very specialized. Some are aimed at specific
domains like peer-to-peer computing simulators (\cite{p2p09-peersim},
\cite{baumgart2009oversim}) or volunteer computing simulators
(\cite{simBA}, \cite{kondo2007simboinc}, \cite{alonso2017combos})

Problems with simulation: often unrealeased simulators, or designed for a
specific project, or short lived (OptorSim) -> This is why Batsim was created.

Simulators specific to platforms: YARNSim, SLURM simulator\footnote{https://github.com/ubccr-slurm-simulator/slurm\_simulator} 

Exemples of papers with custom unreleased simulators: \cite{yabuuchi2019lowlatency} by the same guys who made kubernetes-simulator.\\

When running simulations two primary concerns are accuracy and scalability.
Accuracy is the measure of the bias between the simulated trace of an execution
of an application and its trace as if it were executed on a real system (the
lower it is, the higher the accuracy). Scalability is the ability of the
simulator to compute simulations quickly, or run large scale experiments.\\

list of simulators
\begin{itemize}
	\item SimGrid, GridSim, CloudSim, GroudSim (to cite the most important).
	\item Other simulators in unrelated domains: SimBA (volunteer computing), PeerSim, OverSim (peer to peer), WRENCH (workflows).
	\item HPC simulation: off-line vs on-line
	\item Interconnected networks Simulation: INSEE (environment for interconnected networks), SICOSYS. Aimed to be used with other tools like SIMICS to extend th ecapabilities.
	\item Low level simulation: SIMICS, RSIM and SimOS (multiprocessor systems).
	\item discontinued/old projects: GSSIM, Simbatch
\end{itemize}

Kubernetes simulation: k8-cluster-simulator, joySim.

Related to Batsim in particular: Aléa and Accasim.


\chapter{Integrating the simulator into Kubernetes}

Batsim is able to run simulations of any distributed system, to study any
event-based scheduler that would implement its message protocol. Kubernetes is
a piece of software where all its component, including the scheduler, revolve
around a central API. Everything is asynchronous as the API can be
accessed anytime by any component.

The question that arises is, can we adapt Batsim to make it support Kubernetes
schedulers? Is it possible to implement an adaptive layer between a synchronous
event based simulator like Batsim and a scheduler implemented following the
asynchronous paradigms of APIs?

It will follow that in order to do so, we re-implemented an API following
Kubernetes specifications and intercepted the scheduler's time to
synchronize it with the simulation time. This allows us to run lengthy
workloads in seconds using a scheduler otherwise supposed to rely on ``real''
machine time. We first describe some technical concepts about Kubernetes and
Batsim, and then describe how we re-implemented the API, intercepted the time,
and handled the synchronization of the different times between Batsim and the
scheduler.


\subsubsection{Batkube features}

TODO: Clearly state what Batkube is capable of, and what it does not do. This
helps describing only the parts of the tools we use while leaving the rest for
the reader to look for on official documentation.

\section{Batsim concepts}

A Batsim simulation is divided into two processes: Batsim itself and the
decision process (the scheduler).  As a consequence, Batsim defines its own
messaging protocol to be able to standardize exchanges with the scheduler. This
protocol takes the form of a text based interface that conveys the events
occuring during the simulation.

\begin{figure}[H]
	\centering
	\includegraphics[scale=0.5]{imgs/batsim-sequence-diag.png}
	\captionsource{Exchanges between Batsim and the scheduler.}{https://batsim.readthedocs.io/en/latest/protocol.html}
	\label{fig:bati-seq-diag}
\end{figure}

A Batsim platform is a SimGrid platform, defined in the xml format. A node can
endorse the role of \textit{master}, \textit{compute\_resource} or
\textit{storage}. Here we only consider master nodes which host the decision
process, and compute resources to which we add our custom resource capacities.
These additional fields are \textit{core}\footnote{The core field is already
present in the resource definition, but it is not forwarded to Batsim for
unknown reasons at the time this report is written.} which is the amount of cpu
the node has and \textit{memory} which is memory capacity of the node. Of
course, storage resources can be taken into account in future development of
Batkube. Also, we do not consider the links between the nodes for now because
we do not support parallel jobs. Finally, Batsim proposes an energy model that
we decided to ignore as well.\\

Batsim takes one or several workload as inputs, which are json files containing
jobs definitions. Figure \ref{fig:bat_wl_ex} gives an example of such workload.
The jobs are defined with the following inputs.  First, they are identified by
an \textit{id} which is unique within each workload and are submitted at a time
defined by the \textit{subtime} field. The \textit{res} field states the number
of resources each job requires, although we don't use this field because we
can't specify \textit{which} resource we require. Instead we pass resource
requests as optional parameters. We support the additional fields \textit{cpu}
which has a minimum value of 100m cpu just like Kubernetes cpu
requests\footnote{https://kubernetes.io/docs/concepts/configuration/manage-resources-containers/}
and \textit{memory} which also complies with Kubernetes compute resource
definitions.  Jobs follow a certain \textit{profile} that defines the nature of
the job.  These profile may be as simplistic as \textit{delay} profile which
makes the resource wait for a given amount of time, or describe parallel tasks
using matrices to describe the amount of exchanges between the allocated nodes.
For now we only consider delays to simplify the implementation of the
simulator. Also, because Kubernetes allows the use of multiple
schedulers\footnote{https://kubernetes.io/docs/tasks/extend-kubernetes/configure-multiple-schedulers/},
we support the field \textit{scheduler} which contains the name of the
scheduler the job should be scheduled with.\\

Batsim messaging interface is based on its protocol. Each message is composed
of the field \textit{timestamp} which contains the current simulation time, as
well the field \textit{events} which is a list of events either from Batsim to
the scheduler, or from the scheduler to Batsim. Figure \ref{fig:batmsg_ex}
depicts a standard message sent from the scheduler to Batsim.

Batkube features are limited because we focused on building a working proof
of concept rather than a fully fledged Kubernetes simulator. This is why only
consider a subset of these messages that we present here. More information on
Batsim's protocol is available on Batsim
documentation\footnote{https://batsim.readthedocs.io/en/latest/protocol.html}

TODO: Adrien's view on this section is that it is not necessary and that a link to the doc is sufficient.

\subsubsection{From Batsim to the scheduler}

\paragraph{SIMULATION\_BEGINS}
contains information about the available resources in the cluster, with
Batsim's configuration.

\paragraph{SIMULATION\_ENDS}
is sent at the very end of the simulation: all jobs have finished, and no more
jobs are left in the queues. Batsim exits on this message.

\paragraph{JOB\_SUBMITTED}
notifies the scheduler that a new job has been submitted. It contains
information about the job type, id and specifications. We only consider jobs of
type \textit{delay} to simplify the models. Delay jobs specifications boil down
to the delay length, to which we add resource requests.

\paragraph{JOB\_COMPLETED}
notifies the scheduler that a job has ended, specifying the reason for it. We
only consider situations where all jobs complete correctly. Their state is then
always COMPLETED\_SUCCESSFULLY in our case.

\paragraph{REQUESTED\_CALL}
is an awnser to a CALL\_ME\_LATER event sent by the scheduler.

\subsubsection{From the scheduler to Batsim}

\paragraph{CALL\_ME\_LATER}
is an incentive from the scheduler for Batsim to wake up at a certain
timestamp. When the timestamp is reached in the simulation, Batsim will send a
REQUESTED\_CALL to the scheduler. In our case, this particular exchange will
serve as the base for time synchronisation between the scheduler and Batsim.

\paragraph{EXECUTE\_JOB}
is sent when the scheduler has made a decision. It contains the id of the job
at stake and the id of the resources it has been scheduled to.

\subsubsection{Bidirectional}

\paragraph{NOTIFY}
is used to send some information to the other peer. In our case, we use the
NOTIFY containing no\_more\_static\_job\_to\_submit to determine if the
simulation has ended: knowing that there are no more jobs susceptible to be
scheduled allow us to fast forward to the end of the simulation, thus saving
execution time.\\

Batsim's output takes the form of a csv file containg information about the
jobs executions. Mainly we take interest in their submission time, execution
time and waiting time. Again, a detailed list of Batsim outputs can be found on
the
documentation\footnote{https://batsim.readthedocs.io/en/latest/output-jobs.html}.
During our experimentations with Batkube we interest ourselves in two metrics
that can be computed from this output:
\begin{itemize}
	\item The \textit{makespan}, which is the total length of the
		simulation. It is defined as the timestamp at which the last
		job finishes executing, minus the origin (in this case, zero).
	\item The \textit{mean\_waiting\_time}, which is the mean time the jobs
		spent waiting for a scheduling decision. The waiting time is
		defined by the duration between the submission time and the
		starting time (here the starting time is equivalent to the time
		at which the job was scheduled. We will see later that these
		two times do not correspond in Kubernetes.)
\end{itemize}

These metrics were chosen because in our case, they are representative of the
accuracy of the simulation. Time synchronization (see section
\ref{sec:time-hijack}) may introduce delays in the scheduling decision, thus
increasing the overall makespan and mean waiting time of the simulation.
%Experiments showed that the makespan does not vary much from simulation to
%simulation, but due to Batkube simulations stochastic nature the decisions
%taken by the scheduler may vary from experiment to experiment, thus introducing
%variations in the mean waiting time.

\section{Kubernetes concepts}

Kubernetes is now a large ecosystem to which more than 2 700 developers have
contributed over the years. In this section we present the fundamental concepts
and the components that make up Kubernetes.

\begin{figure}[h]
	\centering
	\includegraphics[scale=0.5]{./imgs/node-overview.png}
	\captionsource{Node overview}{https://kubernetes.io/docs/tutorials/kubernetes-basics/explore/explore-intro/}
	\label{fig:node-overview}
\end{figure}

The basic processing unit of Kubernetes is called a \textbf{pod} which is
composed of one or several containers and volumes\footnote{Because of their
	transient nature, containers can not store data on their own. A volume
	is some storage space on the host machine that can be linked to
containers, in order for them to read and write persistent information.}. The
type of application they contain vary depending on the context: in a web
platform context a pod most often hosts a service or micro-service that must be
available at all times, in opposition to a batch processing context where it
runs an application that is to be executed in a finite amount of time.  Pods
are bundled together in \textbf{nodes} (figure \ref{fig:node-overview}) which
are either physical or virtual machines. They represent another barrier to pass
through to access the outside world which bundles pods under the same network
to facilitate communication between them, and enables the use of proxies to
access the underlying services. A set of nodes is called a \textbf{cluster}
which is the highest abstraction layer in Kubernetes.

Nodes take the idea of containerization further than plain containers by
encapsulating the already encapsulated services.  Each node runs at least one
pod, the \textbf{kubelet}, which is a process responsible for communicating
with the rest of Kubernetes. More precisely, the kubelet communicates with the
\textbf{kube-api-server} which is responsible for the whole cluster. We refer
to this API as the api-server, as it is called within the Kubernetes community.
This API server, as well as the other components of the \textbf{Control Plane}
(figure \ref{fig:kube-components}), can be run on any machine but for
simplicity they are set up on the same machine at start up. This machine is
often called the \textbf{master} node and typically does not run any other
container.

As stated before Kubernetes revolves around its API server which is its central
component. All operations between components go through this REST API. These
operations take various forms like user interactions through the commande line
interface \textbf{kubectl}, scheduling operations or management of cluster data
on \textbf{etcd}. We then decided to re-build the API in order to simulate any
cluster to -- almost -- any Kubernetes scheduler.

TODO: Talk about Kubernetes schedulers. What they are based on, how they communicate to Kubernetes.

\begin{figure}[h]
	\centering
	\includegraphics[width=\textwidth]{./imgs/components-of-kubernetes.png}
	\captionsource{Components of Kubernetes}{https://kubernetes.io/docs/concepts/overview/components/}
	\label{fig:kube-components}
\end{figure}

\section{General architecture of Batkube and its integration with Kubernetes and Batsim}

TODO: move the parts where I explain what we did not do to the appendix

\subsection{Integration with Kubernetes}

In order to adapt Kubernetes schedulers for use with Batsim we need to position
ourselves between the Kubernetes scheduler and the cluster. There are several
options here, at different levels of the cluster. All these are made possible
by the fact that Kubernetes is entirely open source and can be reverse
engineered and modified to suit our needs. First we present the options we
considered but did not choose, and then we explain how we wrote a custom REST
API which acts as a Kubernetes cluster to the scheduler, and as an event based
scheduler to Batsim.

\subsubsection{In between the api and the kubelets}

This is the lowest level option. We position the simulator so as to simulate
just the infrastructure and avoid tampering with Kubernetes resource
management, which is done in their API. This approach would allow us to
effortlessly use any Kubernetes scheduler once their API is supported by
Batkube, and potentially produce the most accurate results. However,
interactions between the kubelets and the API are not documented because the
typical user is not supposed to have to deal with this part of Kubernetes. This
would hinder the development of Batkube because a reverse engineering process
would be required beforehand to understand the intricacies of internal
Kuberenetes exchanges.

\begin{figure}[h]
	\centering
	\includegraphics[width=\textwidth]{imgs/architecture-as-kubelets.png}
	\caption{Mocking the cluster itself.}
	\label{fig:mock_nodes}
\end{figure}

\subsubsection{Custom client-go}

Most Kubernetes schedulers rely on
client-go\footnote{https://github.com/kubernetes/client-go}, which is a Go
client for the api-server. It is a library implementing various tools to help
schedulers converse with the API. By altering this client and patching
schedulers so they use our client instead, we can make it exchange with Batsim
instead of the API. 

\begin{figure}[h]
	\centering
	\includegraphics[scale=0.8]{imgs/custom-go-client.png}
	\caption{Custom Go client to redirect scheduler communications to Batsim}
	\label{fig:custom-go-client}
\end{figure}

Contrary to the kubelets, client-go is a user interface and therefore it is
documented, facilitating reverse engineering of its source code. Still, it
represents thousands of lines of code and altering it to our needs would not be
an easy feat.  The other drawback to this approach is that Batkube would only
support schedulers written in Go and making use of client-go, although this
should not be an issue as the only kubernetes scheduler we could find that does
not rely on client-go is a toy scheduler written in bash \cite{bash-scheduler}.

\subsubsection{Custom API}


Re-implementing the API offers a middle ground between the low level and
undocumented solution of the mock nodes, and the higher level and technically
challenging solution of a client-go fork. Again, there are several options here.

A partial reimplementation of the API would save us the task of building a new
API from the ground up. However this would imply digging deep into the
api-server code in order to understand how the api is organized and what code
we would have to alter. In the end, it is easier to simply build a new API,
since there are tools to help us generate it from its specification.

\begin{figure}[h]
	\centering
	\includegraphics[scale=0.8]{imgs/partial-reimplem.png}
	\caption{Partial reimplementation of the api-server.}
	\label{fig:partial_reimp}
\end{figure}

Furthermore, building a new API allows us to consider only the endpoints we
need and have complete control over the source code. The technically
challenging aspect here is Kubernetes resource management. Indeed, we need to
provide the scheduler with expected informations about the cluster state if we
want to obtain a correct behavior from it, and while the endpoints of the API
are well documented, Kubernetes team did not write lengths about how resources
are managed internally.

\begin{figure}[h]
	\centering
	\includegraphics[width=\textwidth]{imgs/custom-restapi.png}
	\caption{Custom REST API in between the scheduler and Batsim.}
	\label{fig:custom-api}
\end{figure}

\subsection{Architecture of Batkube}

Figure \ref{fig:batkube-architecture} depicts the architecture of Batkube,
which is written in Go.  The central component is the \textbf{broker}. It
handles the messages coming from Batsim and the scheduler while ensuring time
synchronization between them. It is responsible for translating and forwarding
messages between Batsim and the scheduler and orchestrates the synchronization
between the two parties.  \textbf{batsky-go} intercepts the calls to Go
\textit{time} library to ensure the scheduler's time is based on the simulation
time instead of machine time. Time requests are forwarded to Batkube which
replies with the current simulation time.  The complete process to ``hijack''
scheduler time is explained in section \ref{sec:time-hijack}. The \textbf{rest
api} is the reimplementation of the Kubernetes api-server. It ensures the
scheduler gets all the necessary information on the cluster state to make its
scheduling decisions, and it is also the receiver of those decisions. Section
\ref{sec:api} explains how we built this API and a tool to automatically
generate its code from the api-server specification.  \textbf{translate} is a
utility package providing functions to translate Kubernetes resources to Batsim
messages, and \textit{vice versa}.

\begin{figure}[h]
	\centering
	\includegraphics[width=\textwidth]{imgs/batkube-architecture-3-synchro.png}
	\caption{Architecture of Batkube}
	\label{fig:batkube-architecture}
\end{figure}

\section{Building the API} \label{sec:api}

The API of Kubernetes follows the
OpenAPI\footnote{https://www.openapis.org/about} 2 specification which is a
standard for describing APIs. Luckily tools exist to generate such
specification from source code, but also to generate code from a specification.
Since the Kubernetes API specification is available on their
reposotory\footnote{https://github.com/kubernetes/kubernetes/blob/release-1.18/api/openapi-spec/swagger.json},
we were able to use such tools. This allowed us not to implement boiler plate
code by hand and fill the gaps where they needed to be filled, leaving empty
the endpoints we do not need. These endpoints can be dealt with later for
future development of the simulator. For this project we used
go-swagger\footnote{https://github.com/go-swagger/go-swagger/tree/master/examples/stream-server}
to generate our code and the API specification corresponds to the release 1.18
of Kubernetes. One downside of this method is that go-swagger forbids to tamper
with the code of the server itself, although we did not need to during this
project.\\

In order to enable communication between Batsim and the scheduler we need to
translate Batsim messages into Kubernetes resources that can be retrieved by
the scheduler and scheduler decisions to Batsim messages. Batsim Jobs are
simply translated to pods. Jobs do exist in Kubernetes, but they are simply
wrappers around pods: when submitting a job to the (real) api-server, the api
ensures that a pod is created and executed to completion. In our case, we do
not need such intermediate. Compute resources are simply translated to
Kubernetes nodes.

\section{Time interception} \label{sec:time-hijack}

TODO: same remark as earlier, move the C implementation part in the appendix.

Kubernetes schedulers are not event based schedulers. They constantly check on
the cluster state and make decisions accordingly, therefore they are based on
machine time to make their decisions. However, in order to have correct
simulations, the scheduler needs to be synchronized with simulation time. We
then need to intercept all calls to machine time to redirect them to the
simulator. There are two approaches to this: either we patch the C library used
by Go (figure \ref{fig:patch-C}) and compile Go using our library, or we patch
Go's time library (\ref{fig:patch-go}) and then patch the scheduler so it uses
our library instead of the official one.

\begin{figure}[]
	\begin{subfigure}{0.5\textwidth}
		\centering
		\includegraphics[scale=0.8]{imgs/time-hijack-C.png}
		\caption{Option A: patching the C library}
		\label{fig:patch-C}
	\end{subfigure}
	\begin{subfigure}{0.5\textwidth}
		\centering
		\includegraphics[scale=0.75]{imgs/time-hijack-Go.png}
		\caption{Option B: patching Go}
		\label{fig:patch-go}
	\end{subfigure}
	\caption{Different approaches to intercept the scheduler calls to machine time}
	\label{fig:patch-time}
\end{figure}

An attempt was first made for the custom C library, which is the lowest level
solution. Going for the low level solution would truly redirect all calls to
machine time which is something we can not guarantee with the second option, as
we will see in section \ref{sec:patch-scheds}. This approach proved challenging
due to circular dependency issues and was ultimately abandoned. We opted for
the second option which consist of modifying Go source code, which requires
some additional work to patch the schedulers but was actually easier to
implement.

\subsection{Redirection of time requests to Batkube}

The module in charge of this time redirection is called
\textbf{batsky-go}\footnote{https://github.com/oar-team/batsky-go}, which re
implements \texttt{time.Now()} as well as timers and tickers. Timers are
structures that are instantiated with a duration as input, that notify the
caller once this duration is elapsed. Timers can be reset, modified or deleted
after initialization, which make their implementation tricky in a parallel
computing context. Tickers are essentially the same structures as timers,
except they regularly notify the caller with the given period of time instead
of exiting after they fire like timers would.

In order to explain batsky-go algorithms as clearly as possible we need to
provide some context about Go channels. Go allows the user to run multi
threaded code easily thanks to its \textbf{go routines}. These routines allow
the user to run code in parallel whithout requiring any setup by simply calling
\texttt{go func()} where \texttt{func()} is a function.  It creates a new
thread and launches the given code in parallel with the encapsulating function.
Essentially, this creates a child process.  Go \textit{channels} are
\textit{``a typed conduit through which you can send and receive values with
the channel operator, <-''}\footnote{source:
https://tour.golang.org/concurrency/2}

\begin{figure}[]
	\begin{minted}{go}
ch <- v    // Send v to channel ch.
v := <-ch  // Receive from ch, and
           // assign value to v.
   \end{minted}
   \caption{Usage of a Go channel}
   \label{fig:go-channel}
\end{figure}

A channel can be shared amongst several process and allow several pieces of
code run in parallel to share data, without having to implement any mutex. By
default sends and receives on a channel are blocking operations. If the two
lines from figure \ref{fig:go-channel} were to be executed one after the other,
the program would remain stuck on the first line, as it would wait for some
process to receive the data sent on \texttt{ch} which only happens on the
second line.\\

\SetKwInput{KwInput}{Input}
\SetKwInput{KwOutput}{Output}

\begin{algorithm}[]
\DontPrintSemicolon
\KwResult{Current simulation time}
\KwInput{req: requests channel, resMap: response channel map}
\KwOutput{now : simulation time}

\If{requester loop is not running}{
	go runRequesterLoop() \tcc{There can only be one loop runing at a time}
}
id = newUUID()\;
res = newChannel()\;
resMap[id] = res \tcc{A channel is associated with each request}
req $<$- id \tcc{The code blocks here until request is handled}
now = $<$-res \tcc{The code blocks here until response is sent by the requester loop}
return now\;
\caption{Time request (time.now())}
\label{alg:now}
\end{algorithm}

Let us call \texttt{time.Now()} calls \textit{time requests}. Channels enable
us to centralize time requests from multiple simultaneous callers in one place
before forwarding all requests to Batkube. There are two parts to
\texttt{batsky-go}. One is composed of the many requesters, and the other one
is the loop centralizing requests and handling exchanges with Batkube which we
will call ``the main loop''.  Figure \ref{alg:now} gives the new implementation
of time.Now(), which is the ``requester'' part. Each request is identified
using a unique id to keep track of the pending requests that are waiting for a
response. \texttt{resMap} is a thread safe channel map that is shared amongst
the requesters and the main loop containing (id, response channel) key-value
pairs. When a new request is made, a new entry is created on this map and the
requester will wait for a response on the newly created channel, associated
with its id. All requests are sent to a unique channel \texttt{req}. To make a
new request, the requester simply sends its id to this channel.

\begin{algorithm}[]
\DontPrintSemicolon
\KwInput{req: requests channel, resMap: response channel map}
\While{Batkube is not ready} {
	wait\;
}
requests = []request\;
\While{req is not empty} {
	id = $<$- req \tcc{Non blocking receive}
	requests = append(requests, id)\;
}
now = getCurrentTimeFromBatkube()\;
\For{m in requests} {
	resMap[id] $<$-now \tcc{The caller resumes execution upon reception}
}

\caption{Requester loop}
\label{alg:reqLoop}
\end{algorithm}

Each iteration of the main loop first waits for a signal from Batkube
indicating that the broker is ready to handle the time request. That way, we
make sure we don't miss out on any request because the requests accumulate on
the \texttt{req} channel in the mean time, while minimizing potential downtime
because we know Batkube will answer instantly. Once we get the signal that
Batkube is ready all is left to do is retrieve the ids from our requesters,
retrieve the time from the simulation, and answer each individual request.
Figure \ref{fig:time-interception} presents a typical exchange between the
scheduler, batsky-go and Batkube.

\begin{figure}[]
	\centering
	\includegraphics[scale=0.43]{imgs/requester_broker_no_CML.png}
	\caption{Exchange breakdown between batsky-go and Batkube.}
	\label{fig:time-interception}
\end{figure}

\subsection{Patching schedulers} \label{sec:patch-scheds}

We have developed a tool for patching the schedulers, so they use our library
instead of the standard \texttt{time} library. This tool is called
\texttt{batsky-go-installer} and is available on github alongside
Batkube\footnote{github.com/oar-team/batsky-go-installer}.

Our approach replaces all calls to the functions of the \texttt{time} library
while leaving the objects as is to ensure compatibility. In that regard, our
approach makes use of standard \texttt{time} objects instead of redefining new
custom objects, which would break compatibility with the rest of the code. It
makes use of Go Abstract Syntax Tree (AST) to go through the source files and
replace the appropriate symbols, while adding the import to our module whenever
it is necessary. Go \texttt{ast} package make it very convenient to search and
replace symbols in the syntax tree.

\section{Time synchronization}

Synchronization of the time between the scheduler and the simulator is the
critical part of Batkube, because this is where we make compromises between the
speed and the accuracy of the simulation. The scheduler is not event-based and
therefore we can never know for sure when it will send a decision. Therefore we
need to listen to the scheduler as much as we can, but while we do so the
simulation advances slowly. In fact, whenever we listen to the scheduler and
wait for a decision we synchronize machine time and simulation time so the
scheduler gets a time that advances as it would if it was not simulated. This
ensures that the internals of the scheduler function correctly. We could speed
up this time to increase simulation speed and study the effect it has on the
scheduler, but we have to leave this for future work unfortunately. During that
time Batsim is blocked and waits for a message from the decision process. When
we give back priority to Batsim time stops advancing on the scheduler side,
effectively suspending the decision process. At this time Batsim advances
forward in the simulation and replies with new events. Figure
\ref{fig:time_sync} gives a breakdown of the exchanges between Batsim, Batkube
and the scheduler with their associated times.

\begin{figure}[]
	\centering
	\includegraphics[width=0.8\textwidth]{imgs/lignes_de_temps.png}
	\caption{Time sync between the three components. The broker has to take
	into account both machine time and simulation time.}
	\label{fig:time_sync}
\end{figure}

The simulator can be tuned with several parameters which have various effect on
the simulation. We study those effects in section \ref{sec:params-eval} To
limit the time we spend waiting for the scheduler we implemented a timeout
policy. The \textit{timeout value} defines the maximum amount of time we spend
waiting for a scheduler decision: after this amount of time we get back to
Batsim to go forward in the simulation.

%\begin{algorithm}[H]
%	\caption{Synchronization of the time}
%	\label{alg:time-sync}
%	\DontPrintSemicolon
%	\KwInput{simulationTimeout: time.Duration; incrementTimeStep: time.Duration; incrementValue: time.Duration}
%	stopWaitingForMessages := false\;
%	incremented := 0\;
%	\While{!stopWaitingForMessages}{
%		getSchedulerDecisions()\;
%		\If{decisionReceived() or incremented >= simulationTimeout}{
%			stopWaitingForMessages = true
%		}\ElseIf{timeSinceLastIncrement() > incrementTimeStep}{
%			\tcc{this last condition is here to slow down this process and give time for the scheduler to take its decisions}
%			now = now + incrementValue\;
%			incremented = incremented + incrementValue
%		}
%	}
%	addCallMeLater() \tcc{Add a CALL\_ME\_LATER event with timestamp now + incrementValue}
%\end{algorithm}

TODO: explanatory text on this diagram. Explain the different parameters:
timeout, max \& base timestep \& backoff multiplier, min delay, scheduler crash
detection, fast forward on no pending jobs.


\chapter{Evaluation and discussion}

Because SimGrid has already been thoroughly tested and validated, we do not
need to run extensive experiments to validate Batkube simulation models.
Moreover, since we only consider simple delay jobs, validation is not really
necessary. Still, even though the underlying models are sound, Batkube adds a
considerable overhead to Batsim because of the time synchronization between the
simulator and the scheduler. We want to verify to what extent time manipulation
impacts the scheduler behavior, and also that Batkube's fake Kubernetes API
mimics the real API well enough to let the scheduler run as expected.

In the next sections, we present the workloads and platforms we chose to study,
how we conducted experiments on a real cluster, and a study on Batkube's
parameters and their effect on the outputs.

\section{Experiments environment}

The entirety of the experiments are done with the default Kubernetes scheduler
\textbf{kube-scheduler} release \textbf{v1.19.0.rc-4} (commit 382107e6c84).
This choice was made because it was the scheduler used during development, and
because supporting another scheduler would mean more development time which we
could not afford. Still, it is sufficient to experiment with the simulator and
verify the scheduler's behavior in the simulation. All scripts used to run the
experiment, process the workloads and generate the graphs present in this
report -- along with some results -- are available on
\textit{batkube-test}\footnote{github.com/oar-team/batkube-test} repository.

\subsection{Real experimental testbed}

In order to validate the simulator results we then need to compare it against
workloads run on a real cluster. For reproducibility and simplicity sake, we
choose to validate the simulator with an emulated cluster run in
containers.\\

\begin{figure}[H]
	\centering
	\includegraphics[scale=0.7]{./imgs/prot-k3s.png}
	\caption{An emulated experiment.}
	\label{fig:emulated-expe}
\end{figure}

Figure \ref{fig:emulated-expe} illustrates how this is done. First, we create a
k3s cluster run using docker-compose -- \texttt{start-cluster.sh} is a helper
script made for this. Then, a Go script which takes a workload as an input
submits the jobs at the right time and writes the ouputs to csv files -- which
have the same format as Batsim's csv output files. We run each workloads 10
times in order to get statistically meaningful results, except for the
realistic workload which we only run one time because it is already 10 hours
long.

The emulated cluster is limited in terms of variety and capacity. First,
\texttt{start-cluster.sh} only allows the nodes to have the same amount of
available cpu, memory or storage because there was no need for any complex
system for our experiences. Secondly, the maximum amount of cpu, memory or
storage we can make available for each node is capped to the host system
capacities. For example, if the host system possesses 8 cpu cores, the nodes
will have a maximum of 8 cpu available. This will have implications when trying
to run workloads recorded on real systems: either we get to find a workload
that complies with the host system capacity (which is very unlikely), or we
adapt the workload so the jobs requirements do not exceed the host
capabilities (see section \ref{sec:studied-workloads}).

\subsection{Studied workloads} \label{sec:studied-workloads}

We consider three workloads, representing three different situations. The first
two are simplistic and very controlled, and the last one depicts a more
realistic case. In all cases the required resources are only quantified in cpu
only to simplify the study. Note that Batkube does support memory requets, we
just do not wish to add this other layer of complexity to our experiments.

\begin{itemize}
	\item A \textit{burst} workload, consisting in an important amount of
		jobs submitted at once.  200 delays with duration 170s and
		requesting 1 cpu are submitted at the origin.
	\item A \textit{spaced} workload, where jobs of the same nature are
		submitted at regular intervals.  200 delays with duration 170s,
		and requesting 1 cpu are submitted every 10s.
	\item A \textit{realistic} workload, which is extracted from a larger
		trace of a real system.
\end{itemize}

The first two workloads are straight forward and could be generated with the
use of a plain text editor (understand \texttt{vim} and its macros). The third
workload required more processing to be obtained.  

\subsubsection{Standard Workload Format processing}

Batsim provides a tool to translate SWF files to its own json definition. It
also works as a workload preprocessor, although we want to process SWF files
very specifically to suit our needs which is why a custom script was
implemented.

First, a trace in standard
workload format (swf) was obtained on a web
archive\footnote{https://www.cs.huji.ac.il/labs/parallel/workload/logs.html}.
The chosen workload was \texttt{KIT-FH2-2016-1.swf} because it is the most
recent and is relatively lightweight. Secondly,
\texttt{evalys}\footnote{https://github.com/oar-team/evalys} allowed us to
extract a subset of this workload lasting for a given period of time and with a
given mean utilization of the resources. We chose a period of 10h with 80\%
utilization of the resources so as to keep reasonable experiment durations --
Later on we experiment with larger workloads to test out Batkube's limits in
terms of scalability.  The third step is translating this extracted workload to
a \texttt{json} file that can be read by Batsim, which is done with a script
written in Go.

After extracting this subset, we are left off with a workload containing jobs
spanning up to 45h and using up to 24048 cpu (or cpu cores), which is undoable
at our scale on our emulated cluster. We need to trim job durations as well as
cpu usage, as we are limited in cpu by the host machine. This is done during
the translation to the \texttt{json} format. The durations are trimmed down to
a maximum of one hour and the cpu usages are normalized so the maximum amount
of cpu requested equals the amount of cpu available per node on the host
machine. Otherwise, the job would be unschedulable which would not present much
interest.

\subsection{Studied platforms}

The platform used for the first two workloads, \textit{burst} and
\textit{spaced} is composed of 16 nodes each heaving one cpu. For the
\textit{realistic} workload however, we use a single node composed of six cores
for the following reasons.

First, the host machines where the experiments were conducted had six cpu cores
available. This means that if we want to be able to run an emulated cluster
equivalent to this platform we can't exceed six cpu per node.  We use the
maximum amount of available cores in order so as not to obtain too low values
when normalizing the resource requests on the jobs. Indeed, Kubernetes only
allows for a precision of 1 milicpu, so any value bellow that is not considered
a significant number. Normalizing on six cpus instead of one allows us to get
more significant number. Also, only one node gives us a satisfactory overall
resource usage: with more than one node one resource is almost always available
making the scheduling operations trivial.

\section{Study of the simulator parameters}

The simulator has a few parameters that impact the simulation speed and
accuracy. The objective is to study the effects of these parameters on the
simulation to better understand the scheduler behavior when running in
coordination with Batkube.\\

The objective here is to fine tune the parameters in order to find a compromise
between accuracy and scalability. We want to know which combination lead us to
the most stable results, while keeping simulation time as low as possible.

The parameters are:
\begin{itemize}
	\item The \textit{minimum delay} we have to spend waiting for the
		scheduler.
	\item The \textit{timeout} value when waiting for scheduler decisions.
	\item The \textit{maximum simulation time step}, which is the maximum
		amount of time Batsim is allowed to jump forward in time.
\end{itemize}

We first study these parameters one by one by fixing the other parameters to
some other value, then we study what effects these parameters have in respect
to one another, and finally we conduct scalability experiments to test
Batkube's stability and performances on large workloads.

\subsection{Minimum delay}

Earlier in the development of batkube we noticed that not leaving enough time
to the scheduler each cycle lead it to crashes and deadlocks, ultimately
failing the simulation. This time is independent from any decision making we
would receive from it which is why it is called \textit{minimum wait delay}
instead of a plain \textit{timeout} - which is in fact another parameter we
will study later.

For each workload, we compute the crash rate every 5ms, from 0ms to 50ms. Each
point is made by running the simulation 15 times and recording the exit code as
well as the simulation time. The other parameters are: \texttt{timeout=20ms};
\texttt{max-simulation-timestep=20s}. As we will see later those do not offer
acceptable simulation results but they allow us to run prompt simulations, as
accuracy do not concern us here.\\

\begin{figure}
	\centering
	\includegraphics[width=0.45\textwidth]{imgs/min-delay_spaced_200_delay170_crash_old.png}
	\caption{Crash rate of the simulations against minimum delay.}
	\label{fig:min-delay_crash}
\end{figure}

\begin{figure}
	\centering
	\includegraphics[width=\textwidth]{imgs/min-delay_duration.png}
	\caption{Mean duration of the simulations (in case of success) against minimum delay. Error bars show confidence intervals at 95\%.}
	\label{fig:min-delay_duration}
\end{figure}

As we can see on figure \ref{fig:min-delay_crash}, the crash rate decreases
dramatically as soon as the minimum delay reaches a certain threshold, which
here is 10ms.  This crashing issue though was resolved with an update of the
scheduler : the success rate flattens out at 100\% -- or around 100\%. Then,
with earlier versions of the scheduler, the user may have to adjust the minimum
delay in order to run simulation smoothly. 

We also observe on figure \ref{fig:min-delay_duration} -- which was made with
an updated scheduler -- a prompt increase in simulation time from delay value
20ms. This is due to the fact that the \textit{timeout} value is 20ms, which is
reached most of the time because the vast majority of the calls to the
scheduler do not result in a decision making. After this value, we notice a
direct correlation between \textit{minimum delay} increase and simulation time
increase. It follows that the best choice for the \textit{minimum delay} now is
zero, and we will use this value for the rest of the experiments.


\subsection{Timeout}

This value is the maximum amount of time we leave for the scheduler to react. A
\texttt{timeout value} not large enough may lead to inaccuracies in the
simulation: for example, if the scheduler needs 30ms to make a decision upon
reception of a message, and the value of the timeout is 20ms, Batkube will
receive the decision on the next cycle which may happen several dozens of
seconds later (depending on the \textit{maximum simulation time step} value).
On the other hand, a \textit{timeout value} too large will induce longer
simulation times unnecessarily. Indeed, once the simulator was given enough
time to process a message, any time following is spent idling. We want to
measure which \texttt{timeout value} is just enough for the scheduler to be
able to make a response without spending any time idling.

We run each workload with a \texttt{timeout value} ranging from 0ms to 100ms,
with a step of 1ms. Each time we measure the duration of the simulation as well
as the makespan and the mean waiting time. The latter two will enable us to
compare the results against the emulated results in order to estimate the
accuracy of the simulation. The other parameters are set to:
\texttt{min-delay=0ms}, \texttt{max-simulation-timestep=20s}

\begin{figure}
	\begin{subfigure}{.5\textwidth}
		\centering
		\includegraphics[width=\linewidth]{imgs/timeout_duration.png}
		\caption{}
		\label{fig:timeout_duration}
	\end{subfigure}
	\begin{subfigure}{.5\textwidth}
		\centering
		\includegraphics[width=\linewidth]{imgs/timeout_makespan.png}
		\caption{}
		\label{fig:timeout_makespan}
	\end{subfigure}

	\centering
	\begin{subfigure}{.5\textwidth}
		\centering
		\includegraphics[width=\linewidth]{imgs/timeout_mwt.png}
		\caption{}
		\label{fig:timeout_mwt}
	\end{subfigure}
	\caption{Effect of the timeout value on the simulation}
	\label{fig:timeout}
\end{figure}

TODO: redo the simulations for the realistic wl, removing some values of
timeout and adding repetition to show aggregated metrics. (non aggregated ones
are too dispersed)

TODO: \ref{fig:timeout_duration} does not have the same scale on the x axis as the others.

TODO: Gantt charts to show the gaps.

As we expected, a \textit{timeout value} too low results in the scheduler
missing a few cycles each time it wants to communicate a decision making, thus
increasing the makespan and mean waiting time.  As the \textit{timeout}
increases, it reaches a point where the scheduler consistently sends decisions
in the same cycle as the one where it has received the message that triggered
the decision making. After this point though the curves keep decreasing,
showing that the gaps keep receding afterwards. However, the gain in accuracy
is shallow and considering that there is, again, a direct correlation between
the \textit{timeout value} and the simulation time, it is desirable to keep
this value at the limit where the results start to stabilize.  In this case,
according to figure \ref{fig:timeout_makespan}, \texttt{timeout-value=50ms}
seems like a decent compromise between accuracy and scalability.

With such simple workloads and platforms, the decision making time is very low
(it is but a matter of milliseconds), but we can imagine it may reach much
higher values given a bigger platform and more complicated workload.

\subsection{Maximum simulation time step}

Having a high maximum time step value will allow Batsim to jump forward further
in time. This may result in skipping scheduler decisions that could have been
made in the mean time, delaying them to when Batsim decides to wake up. We
expect increasing this value to have an analogous effect to the timeout value:
higher simulation speed, but also decreased accuracy due to gaps (delays) in
the decision process.

To experiment with the maximum time step effect on the results we obtain, we
run each workload with different values of \texttt{max-simulation-timestep}
following a logarithmic scale. The other parameters are fixed to
\texttt{min-delay=0s}, \texttt{timeout=50ms}. Also, the
\texttt{base-simulation-timestep} was lowered to 10ms in order to test lower
values of the maximum timestep (compared to the previous 100ms).

\begin{figure}[]
	\centering
	\includegraphics[scale=0.5]{imgs/max-timestep_duration.png}
	\caption{Effect of maximum timestep on simulation time}
	\label{fig:max_timestep_duration}
\end{figure}

\begin{figure}[]
	\centering
	\includegraphics[scale=0.5]{imgs/max-timestep_mwt.png}
	\caption{Effect of maximum timestep on mean waiting time}
	\label{fig:max_timestep_mwt}
\end{figure}

As expected, the simulation time decreases drastically when the maximum
timestep increases. Still, this value reaches a minimum eventhough the maximum
timestep keeps increasing. This happens because Batsim events are only so far
appart in the simulation, and Batsim will always wake up before the maximum
timestep is reached.

TODO: redo the experiment for the spaced wl and redo the graph for makespan (do
not put all three wl on the same graph because the scale is so different).

\subsection{Parameters inter dependency}

Studying the parameters independently is not enough, we need to study their
impact relatively to each others.

For instance, the \textit{maximum simulation time step} and the
\textit{timeout} value are tightly linked together regarding their effect on
accuracy. Both decreasing \textit{timeout} and increasing \textit{maximum
simulation time step} will increase the amount of delays in the decision making
of the scheduler, but also their length, multiplying the impact on accuracy.

On the other hand, we do not except the \textit{minimum wait delay} to have any
impact other than increasing the simulation stability.

PROTOCOL: We know the simulations are statistically the same by now, and we're
confident enough to run only one simulation per point. Timeout value ranges
from 5ms to 50ms, max timestep ranges from 1s to 100s (logarithmic again).

TODO: Facet graphs: timestep vs timeout vs accuracy vs simulation time. We can
define accuracy as one on the euclidean distance between emulated and simulated
makespan and mean waiting time: \[\frac{1}{\sqrt{(makespan_{sim} -
	makespan_{emu})^2 + (waitingtime_{sim} - waitingtime_{emu})^2}}\]

\section{Validation of the simulator outputs}

TODO: With default parameters (timeout TBD with experiments results, max time step same, min delay 0), compare simulated and emulated results. 

Here : the Gantt charts from evalys.

Study on two metrics : makespan and mean\_waiting\_time. Show the box plots for
simulated and emulated metrics.

Discussion:

Container pull and startup time not accounted for in the simulation.

The scheduler over allocates when it should not, reducing makespan.

The scheduler sometimes does not seem to get update on nodes resource state and
realizes late that some nodes are free (is this still the case? need to plot
gantt charts)

\section{Scalability}

Batkube is by no means scalable in comparison of existing batch schedulers
(TODO: ref to some papers to prove this point). At its current state, it exists
to prove adapting Kubernetes schedulers is possible, leaving optimization for
future works. Also, Kubernetes is not optimized for batch scheduling and usage
in the HPC field is still in its early states.


\chapter{Future work}

TODO: Project into the future. What needs to be fixed, and what needs to be
implemented for Batkube to become a fully-fledged Kubernetes simulator based on
the sound models of SimGrid.


\appendix
\appendixpage
% \addappheadtotoc

\section{Reproducing the experiments}

TODO: organize this part (for now everything is simply copy pasted here) and complete it\\

The command used to run the scheduler is
\begin{minted}{bash}
	./scheduler --kubeconfig=<kubeconfig.yaml>
	--kube-api-content-type=application/json --leader-elect=false
	--scheduler-name=default
\end{minted}
\noindent Only the path to the \texttt{kubeconfig.yaml} changes to either point the the
emulated or simulated cluster.

Batkube is run with \mint{bash}| ./batkube --scheme=http --port=8001 |
\noindent followed by the simulator options.

Batsim is run with option \texttt{enable-compute-sharing}: for a reason
unknown, Kubernetes scheduler tends to over allocate resources in some cases
(especially with smaller jobs) which makes Batsim crash if this option is
disabled. We must allow compute sharing even when it is not expected in order
to capture the scheduler behavior as precisely as possible.\\

Those are the Batkube options that did not vary during the experiments:
\begin{itemize}
	\item \texttt{backoff-multiplier}: 2 (default value)
	\item \texttt{detect-scheduler-deadlock}: true. Obligatory for
		automating experiments
	\item \texttt{fast-forward-on-no-pending-jobs}: the scheduler is not
		susceptible to reschedule running jobs (there is a de-scheduler
		for that) so we might as well fast forward when there is
		nothing to schedule.

\end{itemize}

The option \texttt{scheduler-crash-timeout} did vary between experiments to
make up for odd scheduler crash detections (it was increased up to 30s).
However, it did not have any impact on the results as we do not take into
account simulation time in case of failure.

TODO

Limits: explain how dirty the resource management system is (non
thread safe, stored in memory, little hacks for the resource version) and
briefly write on how it induces problems for the scheduler (over allocating
resources) (we talk about this in the evaluation part).

\begin{figure}
	\begin{minted}{js}
{
  "now": 1024.24,
  "events": [
    {
      "timestamp": 1000,
      "type": "EXECUTE_JOB",
      "data": {
        "job_id": "workload!job_1234",
        "alloc": "1 2 4-8",
      }
    },
    {
      "timestamp": 1012,
      "type": "EXECUTE_JOB",
      "data": {
        "job_id": "workload!job_1235",
        "alloc": "12-100",
      }
    }
  ]
}
\end{minted}
\caption{Example of a Batsim message}
\label{fig:batmsg_ex}
\end{figure}


\begin{figure}
	\begin{minted}{js}
{
    "nb_res": 1,
    "jobs": [
	{"id":"1", "subtime":0, "res": 1, "profile": "delay10"},
	{"id":"2", "subtime":3.4, "res": 1, "profile": "delay10"}
    ],
    "profiles": {
	"delay10": {
	    "type": "delay",
	    "delay": 10,
	    "scheduler": "default",
	    "cpu": "1.5",
	    "memory": "500Mi"
	}
    }
}
	\end{minted}
	\caption{Example of a Batsim workload}
	\label{fig:bat_wl_ex}
\end{figure}

TODO

Explain the first implementation of timer requests that generate call me laters.

\subsection{batsim messages} \label{sec:batmsg}

\subsubsection{From Batsim to the scheduler}

\paragraph{SIMULATION\_BEGINS}
contains information about the available resources in the cluster, with
Batsim's configuration.

\paragraph{SIMULATION\_ENDS}
is sent at the very end of the simulation: all jobs have finished, and no more
jobs are left in the queues. Batsim exits on this message.

\paragraph{JOB\_SUBMITTED}
notifies the scheduler that a new job has been submitted. It contains
information about the job type, id and specifications. We only consider jobs of
type \textit{delay} to simplify the models. Delay jobs specifications boil down
to the delay length, to which we add resource requests.

\paragraph{JOB\_COMPLETED}
notifies the scheduler that a job has ended, specifying the reason for it. We
only consider situations where all jobs complete correctly. Their state is then
always COMPLETED\_SUCCESSFULLY in our case.

\paragraph{REQUESTED\_CALL}
is an awnser to a CALL\_ME\_LATER event sent by the scheduler.

\subsubsection{From the scheduler to Batsim}

\paragraph{CALL\_ME\_LATER}
is an incentive from the scheduler for Batsim to wake up at a certain
timestamp. When the timestamp is reached in the simulation, Batsim will send a
REQUESTED\_CALL to the scheduler. In our case, this particular exchange will
serve as the base for time synchronisation between the scheduler and Batsim.

\paragraph{EXECUTE\_JOB}
is sent when the scheduler has made a decision. It contains the id of the job
at stake and the id of the resources it has been scheduled to.

\subsubsection{Bidirectional}

\paragraph{NOTIFY}
is used to send some information to the other peer. In our case, we use the
NOTIFY containing no\_more\_static\_job\_to\_submit to determine if the
simulation has ended: knowing that there are no more jobs susceptible to be
scheduled allow us to fast forward to the end of the simulation, thus saving
execution time.\\

\subsection{Batsim integration with Kubernetes} \label{sec:imp_levels}

Here are the options that were considered but not chosen.

\subsubsection{In between the api and the kubelets}

This is the lowest level option. We position the simulator so as to simulate
just the infrastructure and avoid tampering with Kubernetes resource
management, which is done in their API. This approach would allow us to
effortlessly use any Kubernetes scheduler once their API is supported by
Batkube, and potentially produce the most accurate results. However,
interactions between the kubelets and the API are not documented because the
typical user is not supposed to have to deal with this part of Kubernetes. This
would hinder the development of Batkube because a reverse engineering process
would be required beforehand to understand the intricacies of internal
Kuberenetes exchanges.

\begin{figure}[h]
	\centering
	\includegraphics[width=\textwidth]{imgs/architecture-as-kubelets.png}
	\caption{Mocking the cluster itself.}
	\label{fig:mock_nodes}
\end{figure}

\subsubsection{Custom client-go}

Most Kubernetes schedulers rely on
client-go\footnote{https://github.com/kubernetes/client-go}, which is a Go
client for the api-server. It is a library implementing various tools to help
schedulers converse with the API. By altering this client and patching
schedulers so they use our client instead, we can make it exchange with Batsim
instead of the API. 

\begin{figure}[h]
	\centering
	\includegraphics[scale=0.8]{imgs/custom-go-client.png}
	\caption{Custom Go client to redirect scheduler communications to Batsim}
	\label{fig:custom-go-client}
\end{figure}

Contrary to the kubelets, client-go is a user interface and therefore it is
documented, facilitating reverse engineering of its source code. Still, it
represents thousands of lines of code and altering it to our needs would not be
an easy feat.  The other drawback to this approach is that Batkube would only
support schedulers written in Go and making use of client-go, although this
should not be an issue as the only kubernetes scheduler we could find that does
not rely on client-go is a toy scheduler written in bash \cite{bash-scheduler}.

\subsubsection{Partial reimplementation of the API}

Re-implementing the API offers a middle ground between the low level and
undocumented solution of the mock nodes, and the higher level and technically
challenging solution of a client-go fork. Again, there are several options here.

A partial reimplementation of the API would save us the task of building a new
API from the ground up. However this would imply digging deep into the
api-server code in order to understand how the api is organized and what code
we would have to alter. In the end, it is easier to simply build a new API,
since there are tools to help us generate it from its specification.

\begin{figure}[h]
	\centering
	\includegraphics[scale=0.8]{imgs/partial-reimplem.png}
	\caption{Partial reimplementation of the api-server.}
	\label{fig:partial_reimp}
\end{figure}

\subsection{Time interception: the C library approach}

An attempt was first made to patch a custom C library, which is the lowest level
solution. Going for the low level solution would truly redirect all calls to
machine time which is something we can not guarantee with the second option, as
we explain in section \ref{sec:patch-scheds}. This approach proved challenging
due to circular dependency issues and was ultimately abandoned. We opted for
the second option which consist of modifying Go source code, which requires
some additional work to patch the schedulers but was actually easier to
implement.

\begin{figure}
	\centering
	\includegraphics[scale=0.5]{imgs/time-hijack-C.png}
	\caption{Option A: patching the C library}
	\label{fig:patch-C}
\end{figure}

\section{Batkube features}

At the time this paper is written, Batkube is in a very early state. The
objective of this work is to establish whether it is possible or not to
implement such interface, and what the time synchronization would imply for the
Kubernetes scheduler.

Therefore, Batkube only supports 

TODO: Clearly state what Batkube is capable of, and what it does not do. This
helps describing only the parts of the tools we use while leaving the rest for
the reader to look for on official documentation.


\backmatter
\printbibliography
\end{document}
